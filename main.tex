\documentclass{article}
\usepackage[utf8]{inputenc}
\usepackage[english]{babel}
\usepackage{amsmath}
\usepackage{amsfonts}

\newtheorem{definition}{Definition}[section]
\newtheorem{remark}{Remark}[section]
\newtheorem{theorem}{Theorem}[section]

\def\slim{\lim_{n \to \infty}  }

\def\R{\mathbb{R}}
\def\iff{\Longleftrightarrow }

\begin{document}



\section{Chapter 1: Tools for Analysis}
\begin{definition}
    For a set of real numbers $S$ that is bounded above, there exist a number such that all
    upper bounds of $S$ are greater than it. This number is the supremum of the set $S$.
\end{definition}
\begin{theorem}
   (Archimedean Principle) 
   
   \begin{enumerate}
       \item Given any number $\epsilon$, there exist a natural number $n$ such that $n>\epsilon$
       \item $(\forall\epsilon\in \mathbb{R}) (\exists n\in \mathbb{N}) \frac{1}{n} < \epsilon$
   \end{enumerate}
\end{theorem}
\begin{theorem}
    The rationals are dense in $\mathbb{R}$
\end{theorem}

\section{Chapter 2: Convergent Sequences}
\begin{definition}
    Given a sequence of numbers ${a_n}$, we say that ${a_n}$ converges to the number $a$
    if $\forall \epsilon>0$, there exists $N\in \mathbb{N}$ such that $\forall n \ge N$
    \[|a_n - a| < \epsilon\]
\end{definition}
Equivalently,
\[ \lim_{n \to \infty} a_n = a \]

\begin{theorem}
    (Properties of Limits) Suppose ${a_n}$ and ${b_n}$ are sequences that converge. Then,
    \begin{enumerate}
        \item (Linearity) $\alpha,\beta \in \mathbb{R}$
        \[ \slim \alpha a_n + \beta b_n = \alpha\slim a_n + \beta\slim b_n\]
        \item (Product Rule)
        \[ \slim a_nb_n = \slim a_n \cdot \slim b_n \]
        \item (Quotient Rule) Suppose also that $b_n \neq 0$ for all $n\in \mathbb{N}$ and
        $\slim b_n \neq 0$. Then,
        \[ \slim \frac{a_n}{b_n} = \frac{\slim a_n}{\slim b_n}\]
    \end{enumerate}
\end{theorem}

\begin{definition}
    A sequence ${a_n}$ is bounded if there exists nonnegative number $M$ such that
    \[ |a_n| \leq M \]
    for all $n$.
\end{definition}

\begin{theorem}
    Every convergent sequence is bounded
\end{theorem}

\begin{theorem}
    A set $S$ is dense in $\R\Longleftrightarrow $ every number $x$ is the limit of a sequence of $S$.
\end{theorem}

\begin{theorem}
    (Comparison Lemma) Suppose $\slim b_n = b$. A sequence $\slim a_n$ converges to a number $a$
    if for some $C\geq 0$ and $N\in \mathbb{N}$,
    \[ |a_n -a| \leq C|b_n -b| \] for $n\geq N$
\end{theorem}

\begin{definition}
    A subset $S$ of $\R$ is closed if ${a_n}$ is a sequence in $S$ that converges in $S$.
\end{definition}

\begin{theorem}
    Every interval $[a,b]$ over $\R$ is closed.
\end{theorem}

\begin{definition}
    A sequence is said to be monotone if
    \[ a_n \geq a_{n-1} \]
    (increasing) or
    \[ a_n \leq a_{n-1} \]
    (decreasing) for all $n$.
\end{definition}

\begin{theorem}
    A monotone sequence converges $\iff$ it is bounded
\end{theorem}

\begin{definition}
    Consider a sequence ${a_n}$. A subsequence is a sequence $b_k$ such that for some
    sequence of strictly increasing natural numbers $n_1 < n_2 < \hdots $,
    \[ b_k = a_{n_k} \]
\end{definition}

\begin{theorem}
    If ${a_n}$ converges to $a$, then every ${a_{n_k}}$ converges to $a$    
\end{theorem}

\begin{theorem}
    (Sequential Compactness)
    Every sequence in interval $[a,b]$ has a subsequence that converges to a number
    in $[a,b]$.
\end{theorem}

\section{Chapter 3: Continuous Functions}

\begin{definition}
    A function $f:D\rightarrow \R$ is said to be continuous at $x_0\in D$ if whenever
    $\{x_n\}$ is a sequence that converges to $x_0$, the sequence $\{f(x_n)\}$ converges
    to $f(x_0)$.

    A function is continuous if it is continuous for all $x_0\in D$.
\end{definition}

\begin{theorem} (Properties of continuity)
    \begin{enumerate}
        \item Products, quotients, and linear combinations of functions $f$ and $g$ are continuous.
        \begin{enumerate}
            \item $f+g:D\rightarrow \mathbb{R}$ is continuous
            \item $f\cdot g:D\rightarrow \mathbb{R}$ is continuous
            \item $f/g:D\rightarrow \mathbb{R}$ is continuous given that g is
            not 0 anywhere in $D$
        \end{enumerate}
        \item Compositions of continuous functions (provided that the domain and images are 
        consistent) are continuous
        \[f\circ g (x)\] is continuous
    \end{enumerate}
\end{theorem}

\begin{proof}
    Properties of convergent sequences.
\end{proof}

\begin{theorem}
    Let $f:[a,b]\rightarrow \mathbb{R}$ be a continuous function. There exist
    a maximum and a minimum of the function.
\end{theorem}

\begin{proof}
    \hfill
    \begin{enumerate}
        \item $f$ is bounded above -- Suppose not bounded. Consider
        a sequence such that $f(x_n) > n$. $x_{n_k}$ converges but
        $f$ continuous.
        \item $\sup f(D)$ is a functional value -- Consider a sequence
        such that $f(x_n) > c-\frac{1}{n}$. $x_{n_k}$ converges to 
        $x_0$ and $f$ continuous so $f(x_0) = c$.
    \end{enumerate}
\end{proof}

\begin{theorem}
    Let $f[a,b]\rightarrow \mathbb{R}$ be a continuous function. For any number
    $c\in(a,b)$, there exist a point $x_0\in [a,b]$ such that
    \[f(x_0) = c\]
\end{theorem}

\begin{proof}
    Bisection method: define $m_n = \frac{a_n+b_n}{2}$. Consider first
    $f(a) < f(b)$. If $c\geq f(m_n)$, set $a_{n+1} = m_n$ (alternatively,
    same with $b_{n+1}$). Now, $a_n\leq a_{n+1} < b_{n+1} \leq b_n$.
    Also, $b_n-a_n = \frac{b-a}{2^{n-1}}$ so it converges to 0. By
    Nested Interval, $\{a_n\}$ and $\{b_n\}$ converge to a point $x_0$.
    $f(x_0) \leq c$ since $f(a_n) \leq c$ and $f(x_0) \geq c$ since 
    $f(b_n) \geq c$, so $f(x_0) = c$.
\end{proof}

\begin{definition}
    An interval $I$ over $\mathbb{R}$ is a set of numbers such that any
    $x<y$ where $x,y\in I$ defines an interval $[x,y]$ in $I$.
\end{definition}

\begin{theorem}
    Let $f:I\rightarrow \mathbb{R}$ function where $I$ is an interval.
    Then, the image of $f$ is also an interval.
\end{theorem}

\begin{proof}
    Definition of interval and Intermediate Value Theorem.
\end{proof}

\begin{theorem}
    Let $f:I\rightarrow \mathbb{R}$ be a function where $I$ is
    an interval. Then, $f$ is continuos if its image $f(I)$ is an
    interval.
\end{theorem}

\begin{definition}
    $f:D\rightarrow R$ is said to converge uniformly in $D$ if whenever
    $\{u_n\}$ and $\{v_n\}$ are two sequences in $D$ such that
    \[\lim_{n\rightarrow \infty} |u_n-v_n| = 0\]
    then
    \[\lim_{n\rightarrow \infty} |f(u_n) - f(v_n)| =0\]
\end{definition}

\begin{definition}
    (Epsilon-Delta Criterion)
    A function is said to satisfy the $\epsilon-\delta$ criterion at a point $x_0$, if
    $\forall \epsilon>0$ there exists $\delta>0$ such that, 
    \[ |f(x) - f(x_0)| < \epsilon \quad \textrm{if} \quad |x-x_0|<\delta\]
    
    ($\epsilon-\delta$ Over a Domain)
    A function is said to satisfy the $\epsilon-\delta$ over a domain if $\forall \epsilon>0$
    there exists $\delta>0$ such that for all $x\in D$,
    \[ |f(u) - f(v)| < \epsilon \quad \textrm{if} \quad |u-v|<\delta\]
\end{definition}

\begin{theorem}
    \hfill
    \begin{enumerate}
        \item A function satisfies the $\epsilon-\delta$ criterion at a point $x_0$
        $\Longleftrightarrow$ it is continuous at $x_0$.
        \item A function satisfies the $\epsilon-\delta$ criterion over a domain 
        $\Longleftrightarrow$ it is uniformly continuous over the domain.
    \end{enumerate}
\end{theorem}
\section{Chapter 9: Sequences and Series of Functions}

\begin{definition}
    A sequence is said to be Cauchy if for all $\epsilon>0$ $\exists N\in \mathbb{N}$
    such that 
    \[ |a_n - a_m| < \epsilon \] for all $n,m \geq N$
\end{definition}

\begin{theorem}
    A sequence converges $\Longleftrightarrow$ it is Cauchy.
\end{theorem}

\begin{proof}
    \hfill
    \begin{enumerate}
        \item ($\Rightarrow$) -- 
        \[ |a_n - a_m| = |a_n -a +a -a_m| \leq |a_n -a| + |a_m - a| \]
        \item ($\Leftarrow$) --
        A Cauchy sequence is bounded since $|a_n| \leq |a_n| + |a_n-a_N| = |a_n| + 1$
        where $N$ is chosen with Cauchy assumption. By sequential compactness, $a_{n_k}$ converges to
        a number a. Then there exist $N'=\max\{N_{Cauchy},N_{sub}\}$ such that for
        $n\geq N'$,
        \[ |a_n -a | \leq |a_n-a_{n_k}| + |a - a_{n_k}| \]
    \end{enumerate}
\end{proof}

\def\series{\sum_{i=1}^\infty a_i}

\begin{theorem}
    (Necessary condition for Series Convergence)
    Suppose $\series$ converges. Then, $\slim a_n = 0$.
\end{theorem}

\begin{theorem}
    Suppose $\{a_n\}$ is nonnegative. $\series$ converges $\Longleftrightarrow$
    $S_n$ is bounded
\end{theorem}

\begin{theorem}
    (Comparison Test)
    Suppose $\{a_n\}$ and $\{b_n\}$ are two sequences. And $a_n\leq b_n$. Then,
    \begin{enumerate}
        \item $\series$ converges if $\sum_{i=1}^\infty b_i$ converges.
        (Example: $\frac{1}{k2^k} \leq \frac{1}{2^k}$)
        \item $\series$ diverges only if $\sum_{i=1}^\infty b_i$ diverges.
        (Example: $\frac{1}{\sqrt{k}}\geq \frac{1}{k}$)
    \end{enumerate}
\end{theorem}

\begin{theorem}
    (Integral Test)
    Suppose a sequence $\left\{ a_k \right\}$ and a function $f$ that has the property
    \[ f(k) = a_k \]  and is continuous and montonically decreasing.
    for all $k$. Then, $\left\{ a_k \right\}$ converges $\Longleftrightarrow$
    the sequence $\{\int_1^n f(x)\, dx\}$ is bounded. 
\end{theorem}

\begin{theorem}
    ($p$-Test)
    For a positive number p, the series
    \[ \sum_{i=1}^\infty \frac{1}{k^p} \]
    converges if and only if $p>1$
\end{theorem}

\begin{theorem}
    (Alternating Series)
    Suppose $\left\{ a_k \right\}$ is a nonnegative sequence, monotonically
    decreasing, and converges to 0. The series
    \[ \sum_{i=1}^\infty (-1)^{k+1}a_k \]
    converges.
\end{theorem}

\begin{theorem}
    If $\sum_{i=1}^\infty |a_i|$ converges, then $\series$ converges.
\end{theorem}

\begin{theorem}
    Given the series $\series$. Suppose there exists $N\in \mathbb{N}$ such that 
    for all $n\geq N$,
    \[ |a_{n+1}|\leq r|a_n| \quad \textrm{for} \quad 0\leq r < 1 \]
    Then, $\series$ converges absolutely.
\end{theorem}

\begin{proof}
    For $n\geq N$,
    \[ |a_1| + |a_2| + \hdots + |a_{N+k}| \leq |a_1| + |a_2| + \hdots + |a_N|\frac{1-r^{k+1}}{1-r} \]
    so the partial sum is bounded for every index $n$.
\end{proof}

\begin{theorem}
    (Ratio Test)
    Suppose for the sequence $\series$,
    \[ \slim \frac{|a_{n+1}|}{|a_n|} = l\]
    Then, $\series$ converges absolutely if $l < 1$ and diverges if $l > 1$.
\end{theorem}

\subsection{Sequence of Functions}

\begin{theorem}
    (Differentiability of Limit Function) \\
    Suppose $\{f_n:I\rightarrow \R\}$ is a sequence of continuously differentiable
    functions and $\{f_n'\}$ the derived sequence. If
    \begin{enumerate}
        \item $\{f_n:I\rightarrow \R\}$ converges pointwise to $f$
        \item $\{f_n'\}$ converges uniformly to a function $g$
    \end{enumerate}
    Then,
    \[ f'(x) = g(x) \quad \forall x\in I\]
\end{theorem}

\end{document}