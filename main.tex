\documentclass{article}
\usepackage[utf8]{inputenc}
\usepackage[english]{babel}
\usepackage{amsmath}
\usepackage{amsfonts}

\newtheorem{definition}{Definition}[section]
\newtheorem{remark}{Remark}[section]
\newtheorem{theorem}{Theorem}[section]

\def\slim{\lim_{n \to \infty}  }

\def\R{\mathbb{R}}
\def\iff{\Longleftrightarrow }

\begin{document}



\section{Chapter 1: Tools for Analysis}
\begin{definition}
    For a set of real numbers $S$ that is bounded above, there exist a number such that all
    upper bounds of $S$ are greater than it. This number is the supremum of the set $S$.
\end{definition}
\begin{theorem}
   (Archimedean Principle) 
   
   \begin{enumerate}
       \item Given any number $\epsilon$, there exist a natural number $n$ such that $n>\epsilon$
       \item $(\forall\epsilon\in \mathbb{R}) (\exists n\in \mathbb{N}) \frac{1}{n} < \epsilon$
   \end{enumerate}
\end{theorem}
\begin{theorem}
    The rationals are dense in $\mathbb{R}$
\end{theorem}

\newpage
\section{Chapter 2: Convergent Sequences}
\begin{definition}
    Given a sequence of numbers ${a_n}$, we say that ${a_n}$ converges to the number $a$
    if $\forall \epsilon>0$, there exists $N\in \mathbb{N}$ such that $\forall n \ge N$
    \[|a_n - a| < \epsilon\]
\end{definition}
Equivalently,
\[ \lim_{n \to \infty} a_n = a \]

\begin{theorem}
    (Properties of Limits) Suppose ${a_n}$ and ${b_n}$ are sequences that converge. Then,
    \begin{enumerate}
        \item (Linearity) $\alpha,\beta \in \mathbb{R}$
        \[ \slim \alpha a_n + \beta b_n = \alpha\slim a_n + \beta\slim b_n\]
        \item (Product Rule)
        \[ \slim a_nb_n = \slim a_n \cdot \slim b_n \]
        \item (Quotient Rule) Suppose also that $b_n \neq 0$ for all $n\in \mathbb{N}$ and
        $\slim b_n \neq 0$. Then,
        \[ \slim \frac{a_n}{b_n} = \frac{\slim a_n}{\slim b_n}\]
    \end{enumerate}
\end{theorem}

\begin{definition}
    A sequence ${a_n}$ is bounded if there exists nonnegative number $M$ such that
    \[ |a_n| \leq M \]
    for all $n$.
\end{definition}

\begin{theorem}
    Every convergent sequence is bounded
\end{theorem}

\begin{theorem}
    A set $S$ is dense in $\R\Longleftrightarrow $ every number $x$ is the limit of a sequence of $S$.
\end{theorem}

\begin{theorem}
    (Comparison Lemma) Suppose $\slim b_n = b$. A sequence $\slim a_n$ converges to a number $a$
    if for some $C\geq 0$ and $N\in \mathbb{N}$,
    \[ |a_n -a| \leq C|b_n -b| \] for $n\geq N$
\end{theorem}

\begin{definition}
    A subset $S$ of $\R$ is closed if ${a_n}$ is a sequence in $S$ that converges in $S$.
\end{definition}

\begin{theorem}
    Every interval $[a,b]$ over $\R$ is closed.
\end{theorem}

\begin{definition}
    A sequence is said to be monotone if
    \[ a_n \geq a_{n-1} \]
    (increasing) or
    \[ a_n \leq a_{n-1} \]
    (decreasing) for all $n$.
\end{definition}

\begin{theorem}
    A monotone sequence converges $\iff$ it is bounded
\end{theorem}

\begin{definition}
    Consider a sequence ${a_n}$. A subsequence is a sequence $b_k$ such that for some
    sequence of strictly increasing natural numbers $n_1 < n_2 < \hdots $,
    \[ b_k = a_{n_k} \]
\end{definition}

\begin{theorem}
    If ${a_n}$ converges to $a$, then every ${a_{n_k}}$ converges to $a$    
\end{theorem}

\begin{theorem}
    (Sequential Compactness)
    Every sequence in interval $[a,b]$ has a subsequence that converges to a number
    in $[a,b]$.
\end{theorem}

\newpage
\section{Chapter 3: Continuous Functions}

\begin{definition}
    A function $f:D\rightarrow \R$ is said to be continuous at $x_0\in D$ if whenever
    $\{x_n\}$ is a sequence that converges to $x_0$, the sequence $\{f(x_n)\}$ converges
    to $f(x_0)$.

    A function is continuous if it is continuous for all $x_0\in D$.
\end{definition}

\begin{theorem} (Properties of continuity)
    \begin{enumerate}
        \item Products, quotients, and linear combinations of functions $f$ and $g$ are continuous.
        \item Compositions of continuous functions (provided that the domain and images are 
        consistent) are continuous
    \end{enumerate}
\end{theorem}

\begin{definition}
    (Epsilon-Delta Convergence)
\end{definition}



\end{document}